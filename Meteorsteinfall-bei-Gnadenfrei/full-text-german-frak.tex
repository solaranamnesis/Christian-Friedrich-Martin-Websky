\documentclass[a4paper, 11pt, oneside]{article}
\usepackage[utf8]{inputenc}
\usepackage[T1]{fontenc}
\usepackage[ngerman]{babel}
\usepackage{yfonts}
%\usepackage{fbb} %Derived from Cardo, provides a Bembo-like font family in otf and pfb format plus LaTeX font support files
\usepackage{booktabs}
\setlength{\emergencystretch}{15pt}
\usepackage{fancyhdr}
\usepackage{graphicx}
\usepackage{microtype}
\graphicspath{ {./} }
\begin{document}
\frakfamily
\begin{titlepage} % Suppresses headers and footers on the title page
	\centering % Centre everything on the title page
	%\scshape % Use small caps for all text on the title page

	%------------------------------------------------
	%	Title
	%------------------------------------------------
	
	\rule{\textwidth}{1.6pt}\vspace*{-\baselineskip}\vspace*{2pt} % Thick horizontal rule
	\rule{\textwidth}{0.4pt} % Thin horizontal rule
	
	\vspace{1.5\baselineskip} % Whitespace above the title
	
	{\scshape\LARGE Bericht über den Meteorsteinfall}
	
	\vspace{1\baselineskip} % Whitespace after the title block

	{\scshape\LARGE bei Gnadenfrei am 17. Mai 1879.}

	\vspace{1.5\baselineskip} % Whitespace above the title

	\rule{\textwidth}{0.4pt}\vspace*{-\baselineskip}\vspace{3.2pt} % Thin horizontal rule
	\rule{\textwidth}{1.6pt} % Thick horizontal rule
	
	\vspace{1\baselineskip} % Whitespace after the title block
	
	%------------------------------------------------
	%	Subtitle
	%------------------------------------------------
	
	{\scshape Websky} % Subtitle or further description
	
	\vspace*{1\baselineskip} % Whitespace under the subtitle
	
    {\scshape\small Von J. G. Galle und A. von Lasaulx.\\ Mit 1 Tafel.} % Subtitle or further description
    
	%------------------------------------------------
	%	Editor(s)
	%------------------------------------------------
    \vspace*{\fill}

	\vspace{1\baselineskip}

	{\small\scshape 31. Juli 1879.}
	
	\vspace{0.5\baselineskip} % Whitespace after the title block

    \scshape Internet Archive Online Edition  % Publication year
	
	{\scshape\small Namensnennung Nicht-kommerziell Weitergabe unter gleichen Bedingungen 4.0 International} % Publisher
\end{titlepage}
\setlength{\parskip}{1mm plus1mm minus1mm}
\clearpage
\tableofcontents
\clearpage
\section{\frakfamily{Hr. Websky legte vor: Bericht über den Meteorsteinfall bei Gnadenfrei am 17. Mai 1879.}}
\paragraph{}
Die erste Nachricht von dem am 17. Mai d. J. Nachmittags gegen 4 Uhr bei Gnadenfrei zwischen Reichenbach und Frankenstein in Schlesischen Zeitung vom 20. Mai enthaltene Mitteilung des Hrn. Grafen L. Pfeil in Gnadenfrei. Es fügte sich besonders günstig, dass schon am nächsten Tage nach dem Falle, am 18. Mai, diesem wissenschaftlichen Kenner der Beschaffenheit und des Wertes derartiger Funde die Nachricht über dieses Ereignis zuging und durch die höchst dankenswerten Bemühungen und Vermittlungen desselben der größte Teil der Stücke für die wissenschaftlichen Zwecke erhalten, sowie über den Verlauf des Niederfalles und die denselben begleitenden physikalischen Erscheinungen sofort die wesentlichsten Erkundigungen eingezogen werden konnten. In Folge der vom Hrn. Grafen Pfeil in der Schlesischen Zeitung vom 20. Mai mitgeteilten ersten Nachricht und demnächstiger brieflicher Korrespondenzen nahmen die unterzeichneten Berichterstatter, Prof. Galle und Prof. von Lasaulx, Anlass, am 24. und 25. Mai noch persönlich nach dem Orte des Falles hinzureisen, um einerseits über die kosmische und physikalische Seite des Phänomens die etwa noch möglichen weiteren Erkundigungen einzuziehen, andererseits um weitere Erwerbungen dort noch vorhandener Meteoritenstücke für die hiesigen Sammlungen einzuleiten, nachdem Hr. Graf Pfeil schon unter dem 20. Mai mehrere Stücke hierher zu senden die Güte gehabt hatte.

Teils nach den Mitteilungen von Graf Pfeil, teils nach den am 24. und 25. von dem erstgenannten Berichterstatter auch noch persönlich bei den Findern der Steine und andern Personen in der Umgegend von Gnadenfrei eingezogenen Erkundigungen ergab sich über den Verlauf des Steinfalles Folgendes. Es sind bisher zwei Steine, ein größerer, etwa 1 Kilogramm schwer, südlich von Gnadenfrei in der Richtung nach dem Dorfe Kleutsch zu, und ein etwas kleinerer, nordöstlich in dem Dorfe Schobergrund, gefunden worden. Von ersterem wurde das Niederfallen in einer Entfernung von etwa 64 Schritten oder 50 m. (wie am 24. an Ort und Stelle nachträglich ermittelt wurde) wirklich beobachtet. Die Schuhmacherfrau Pauline Neumann aus Ober-Peilau, auf dem Felde zwischen diesem Orte und Kleutsch östlich von dem sogenannten Mittelberge mit einem Schubkarren nach Kleutsch hin sich bewegend, wurde bei leicht und größtenteils bewölktem Himmel plötzlich durch einen heftigen Knall wie einen Kanonenschuss erschreckt, dem ein Knattern wie Kleingewehrfeuer folgte, so dass sie in südöstlicher Richtung in dem Walde östlich von Kleutsch Jäger vermutete. Weiter gehend hörte sie kurz nachher ein starkes Summen oder Sausen und sah, sich nach rechts umwendend, mit einem dumpfen Schläge etwas in den Acker fallen und den lockeren schwarzen Boden aufwerfen. Sie rief eine andere in einiger Entfernung auf dem Felde arbeitende Frau herbei, um mit ihr gemeinschaftlich nachzusehen, was dort wohl heruntergefallen sein könne. Auf diese Weise wurde aus dem etwa 1 Fuß tiefen senkrechten Rinde überzogener Stein von ihnen ausgegraben, welcher kalt war und in einem nahen Graben abgewaschen wurde. Der Stein wurde mitgenommen und von der zweiten Frau auch ihrem Manne gezeigt, der ein Stück abschlug und durch den mittels einer dritten Person am folgenden Tage die Nachricht von dem Falle zur Kenntnis des Grafen Pfeil gelangte. Bei dem nochmaligen Aufsuchen des Niederfallpunktes und des noch vorhandenen Loches am 24. Mai durch den Berichterstatter in Begleitung des Hrn. Grafen Pfeil und der P. Neumann wurde der Weg nochmals durchschritten, den dieselbe von dem Knalle bis zu dem Momente des Niederfalles zurückgelegt zu haben glaubte und eine Zeitdauer von etwa 70 Sekunden ermittelt, um welche der Niederfall später erfolgt sei als der Knall. Das Loch erschien am 24. Mai wegen des Ausgrabens etwas ausgeweitet, jedoch noch immer senkrecht. Die Richtung des niederfallenden Steines selbst erschien der Beobachterin entweder senkrecht oder vielleicht ein wenig von rechts oben nach links unten, was einer Richtung etwas mehr von Osten als von Westen her entsprechen würde. --- Das Dorf Schobergrund, wo der zweite Stein gefallen war, wurde von dieser Stelle aus Abends 6 h. 54 m. um 110° rechts von dem Azimut der Sonne geschätzt. Da nun das Sonnen-Azimut für diese Zeit zu 114° sich berechnet, so befindet sich der Niederfallpunkt des zweiten Steines von dem des ersten aus gesehen in der Azimutal-Richtung 224° oder sehr genau in NO. Dasselbe ergibt die neue Liebenowsche Spezialkarte der Grafschaft Glatz, wonach ferner der Abstand der Fallpunkte der beiden Steine voneinander sehr nahe auf drei Kilometer zu setzen ist.

Am 25. vormittags wurden die Erkundigungen fortgesetzt. Der Berichterstatter begab sich zunächst nach dem Dorfe Kleutsch, wo der Dorfschmidt zugleich bei dem Hören des Donnerns am Himmel Streifen gesehen haben sollte. Derselbe wurde indes nicht angetroffen und es muss als etwas sehr Fragliches dahingestellt bleiben, ob bei dem fast ganz bewölkten Himmel nicht eine einfache Verwechselung mit gewöhnlichen Wolkenstreifen stattgefunden hat, da von den mit der Entzündung verbundenen Vorgängen schwerlich irgend etwas unterhalb der Wolkendecke sichtbar gewesen ist. Die Frau des Schmidts befand sich im Besitz einiger kleiner Brocken des zweiten (Schobergrunder) Steines, und da ein etwas größeres Stück davon, wie in Erfahrung gebracht wurde, in den Besitz des Inspektors des Dominiums Hrn. Kühn gekommen war, so begab sich der Berichterstatter auch zu diesem. Derselbe hatte die Gefälligkeit, mehrere Leute des Dorfes kommen zu lassen, welche unweit Kleutsch die Schallerscheinungen gehört hatten. Eine Frau kam von Gnadenfrei und hatte noch nicht den Fußweg, welcher vom Mittelberge her nach der Kleutscher Chaussee führt, verlassen, als sie kurz vor der Chaussee den Donner hörte, scheinbar hinter ihr vom Questenberge kommend; sie eilte furchtsam, ohne sich viel umzusehen, nach Kleutsch zu. Der Schmiedemeister des Dominiums Völkel und noch ein älterer Mann befanden sich nahe bei Kleutsch auf der zum Bahnhofe Gnadenfrei führenden Straße unweit der Windmühle. Beide waren mit Zerschlagen von Steinen beschäftigt. Sie hörten einen heftigen Knall, darauf ein lange anhaltendes Sausen, wie das Summen von Telegraphendrähten im Winde, dann ein längeres Geknatter, so dass sie ein Herankommen von Militär von Schobergrund her vermuteten. Es wurde dies stärker, ließ dann nach und verzog sich nach dem Mittelberg und nach Reichenbach hinwärts. Sie vermuteten, es sei etwas in der Gegend des ehemaligen großen Teiches niedergefallen, unweit welches Terrains in der Tat der größere Stein gefallen ist. Dass das eigentümliche Summen noch vor und zwischen dem Geknatter stattgefunden habe, wurde bestimmt von ihnen behauptet, entgegen der ebenso bestimmten Aussage der P. Neumann, die das Summen des niederfallenden Steines erst nach den Knallen hörte. Es sind diese widersprechenden Aussagen schwer zu vereinigen und könnte nur etwa angenommen werden, dass eine Anzahl kleinerer Steine des Schwarmes schon etwas früher als der größere Stein zur Erdoberfläche unweit Kleutsch niedergegangen wäre, ohne gefunden zu sein, oder dass das Sausen des größeren Steines 2 Kilometer weit hörbar gewesen wäre, welches letztere schwer anzunehmen ist und zur Erklärung der Zeitdifferenz noch immer nicht genügen würde. Bei den Widersprüchen dieser Angaben unter sich dürfte es sich rechtfertigen, von einer genaueren Diskussion derselben und etwaigen Schlüssen daraus auf die Höhe des Hemmungspunktes der Steine in diesem Falle abzusehen.

Von Kleutsch wurde der Weg zurück und Ober-Peilau durchschneidend demnächst nach Schobergrund genommen, wo etwas später noch an demselben Tage ein zweiter etwas kleinerer Stein gefunden worden war, ohne dass man den Niederfall selbst beobachtet hatte. Unmittelbar an das Dorf Schobergrund schließt sich die dazu gehörende an einer Anhöhe sich hinaufziehende Kolonie Sadebeckshöhe an, wo nahe an einem der ersten Häuser, dem des Arbeiters und Steinschleifers Gagsch, und zwar auf einem nordöstlich daran sich anschließenden kleinen Gerstenfelde der Stein gefallen war. Der Nachbar desselben, Günther, stand etwa fünf Minuten Weges weiter nordöstlich am Abhange des Berges und hörte einen Schall wie Kanonendonner und wie das Sprengen von Steinen. Ein Arbeiter mit Vornamen Thomas aus dem nahen Gasthofe zum grünen Thale in Schobergrund befand sich auf der Strasse vor dem Gehöfte des Gagsch, hörte einen gewaltigen Knall und ging darauf etwa eine Minute (?) weiter, als er plötzlich ein eigentümliches Summen, wie von dem nahen Schobergrund (W.) her, hörte und meinte, dass an zwei Stellen im Thale westlich und dann auch östlich etwas niedergefallen sein müsse. Auch folgte ein Geknatter, jedoch erscheinen die Zeitangaben an sich und in Betreff der Reihenfolge unzuverlässig, Die Tochter und die Frau des Wirtes Adam in dem genannten Gasthofe hatten den starken Donner gleichfalls gehört. Erstere behauptete bestimmt, drei starke Schläge gehört zu haben, worauf ein längeres Rollen und Geknatter folgte. Einige Zeit nachher besah der Arbeiter Gagsch sein mit jung aussprossender Gerste bestandenes kleines Ackerfeld hinter dem Hause und bemerkte, den Rain entlang gehend, in drei Schritt Entfernung ein Loch im Acker, wovon er glaubte, dass es vielleicht der Hund aufgewühlt habe. Dasselbe war senkrecht 6---8 Zoll tief und in demselben erblickte er den nahe 14 Pfund schweren Stein, der hiernach, in Verbindung mit den vorher bekannt gewordenen Erzählungen der Nachbarn, als möglicherweise vom Himmel gefallen betrachtet, nun aber auch leider in viele Stücke zerschlagen wurde, welche teils im Orte selbst, teils nach Kleutsch und sonst verteilt und verschleppt wurden. Nur durch die Bemühungen des Hrn. Grafen Pfeil, welcher auch hiervon erfuhr, wurden noch mehrere Stücke wiederum zusammengebracht, sowie auch die im Besitze des Finders Gagsch und des Arbeiters Thomas verbliebenen Stücke von dem Berichterstatter für die Breslauer Sammlungen erworben wurden. Das etwa 8 Zoll tiefe senkrechte Loch war am 25. Mai, acht Tage nach dem Falle, noch unverändert vorhanden, und die Natur des Steines gestattete über den Ursprung desselben und über die Zusammengehörigkeit mit dem auf der andern Seite von Gnadenfrei gefallenen Steine keinen Zweifel.

Ob in der Nähe von Schobergrund etwa noch ein dritter größerer Stein gefallen sei, wie dies der Aussage des Thomas und auch der Angabe der Tochter des Wirtes Adam in Betreff des Hörens von drei stärkeren Knallen entsprechen würde, hat bisher nicht in Erfahrung gebracht werden können. Die dem Berichterstatter bekannt gewordenen Nachrichten geben auch keine völlige Gewissheit über die Richtung, von welcher her die Meteoriten gekommen sein können. Da jedoch bei dem Niederfallen eines in schräger Richtung aus dem Weltraume herabkommenden Steinregens die Steine im Allgemeinen nach ihrer Größe sich ordnen und die größeren Stücke weiter voran liegen, so ist mutmaßlich die Verbindungslinie der beiden Orte des Niederfalles dieser zwei großen Steine (SW.---NO.) die weiteste Grenze der sonst etwa noch gefallenen Steine nach NW. hin, und es dürfte wahrscheinlicher sein, dass die Steine von SO., als dass sie von NW. kamen, da nach SO. hin mehr Nachrichten über die Schallerscheinungen bekannt geworden sind, sowohl was den aus der Höhe kommenden Donner, als was das den Niederfall begleitende Sausen betrifft. Jene einem Geschützdonner ähnelnden Schallerscheinungen wurden auch in dem zwei Meilen westlich gelegenen Hausdorf in der Grafschaft Glatz und südlich noch jenseits des Zobtens bei fünf Meilen Entfernung in der Nähe von Canth wahrgenommen.

Die Bewölkung des Himmels und die unsicheren und teilweise einander widersprechenden Angaben über den Verlauf der Schallerscheinungen lassen in dem vorliegenden Falle eine weitere Förderung der Aufschlüsse über die physikalischen Vorgänge bei dem Eintreten der Meteoritenschwärme in die Erdatmosphäre kaum noch erwarten, wie solche dem Berichterstatter früher bei seiner Berechnung des Pultusker Meteorsteinfalles sich dargeboten haben und in den Schriften der Schlesischen Gesellschaft vom Jahre 1368 veröffentlicht sind. Obwohl aber mehrere seitdem beobachtete Meteore die Prinzipien der damaligen Ermittlungen lediglich bestätigt haben, so erscheint doch namentlich eine fortgesetzte sorgfältige Sammlung und Prüfung von Beobachtungen über die Schallerscheinungen auch noch weiterhin von Wert: da besonders die Fragen wegen des Luftwiderstandes während des Herabfallens der Steine von ihrem Hemmungspunkte aus einer genaueren Lösung noch harren und bis zu einem gewissen Grade auch in Betreff der Entstehung und der Geschwindigkeit des Schalles in den sehr hohen Regionen der Atmosphäre noch Bedenken obwalten können.

Beide von diesem Falle aufgefundenen Steine waren, als sie aufgehoben wurden, rundum mit dunkler Schmelzrinde umgeben. Sie wurden dann aber leider sogleich in Stücke zerschlagen. Von dem zu Gnadenfrei niedergegangenen Steine blieb jedoch ein großes Stück im Gewichte von 751,86 gr. erhalten und kam mit sieben kleineren Stücken, deren Gesamtgewicht 150,86 gr. betrug, in unseren Besitz. An das große Stück, das in Fig. 1---3 dargestellt ist, passten noch drei Stücke mit den Bruchflächen genau. In Fig. 3 sind links an den Buchstaben i k l zwei dieser Stücke wieder angefügt, während die Fig. 1 und 2 nur das größere Stück darstellen. Aus den übrigen noch vorhandenen kleineren Stücken lässt sich der noch fehlende Teil sehr gut ergänzen. Man sieht das in Fig. 3 an der nach vorne liegenden Bruchfläche, schon ohne die Anwesenheit der kleineren Stücke. Jedenfalls kann das Fehlende nicht mehr betragen haben, als die Summe der außer dem großen Stücke noch in unserem Besitz befindlichen Teile, also etwa höchstens 150 gr. Das ursprüngliche Gesamtgewicht dieses Steines würde hiernach 1032 gr. oder rund 1 Kilo betragen haben. Die Form des Gnadenfreier Steines ist nicht besonders auffallend; sie stimmt mit manchen der in unserer Sammlung befindlichen Steinen von Pultusk ganz überein. Nach der Dünne der Schmelzrinde und der Beschaffenheit charakterisiert sich die in Fig. 3 nach oben liegende Wölbung und speziell der zwischen den Buchstaben a und k liegende Teil als die Brustfläche. Alle übrigen Flächen, am ausgezeichnetsten die Rückenfläche (in Fig. 2 die über d dargestellte, in Fig. 3 nach unten liegend), zeigen die fingerartigen Eindrücke, herrührend von dem Abspringen und ungleichmäßigen Anschmelzen einzelner Teile der Oberfläche. Die ganze Oberfläche der Schmelzrinde ist mit sehr feinen, wellig verlaufenden Runzeln bedeckt. Zahlreiche kleine, meist rundliche Höcker rühren von den durch die Rinde hervortretenden Eisenkörnern oder auch chondritischen Kügelchen her. Unter der dünnen Rinde der Brustfläche treten diese besonders hervor. Zahlreiche feine Risse durchsetzen die Schmelzrinde; sie sind wohl alle erst beim Einschlagen in den Boden oder beim Zerschlagen des Steines entstanden; nirgendwo hat auf denselben ein Eindringen der Schmelzhülle ins Innere stattgefunden.

Auf der Rückenfläche von dem Punkte bei b Fig. 2 bis zur äußersten ergänzten Spitze, die in der Figur fehlt, beträgt die Länge 15 cm., in der dazu senkrechten Richtung, also etwa parallel der Kante b d (Fig. 2) über die Mitte der Fläche 9 cm. Die Dicke beträgt bei b Fig. 1 3 cm., bei a 6 cm.

Von dem zweiten zu Schobergrund niedergegangenen Steine sind zehn Stücke in unseren Besitz gelangt. Das größte derselben wiegt nur 57,285 gr., (ein anderes 54,15 gr.), das kleinste 3,54 gr. Das Gesamtgewicht derselben beträgt 260,4 gr. Außerdem besitzt die Realschule zu Reichenbach ein Stück im Gewichte von 29,78 gr., ein weiteres Stück befindet sich, im Besitz des Hrn. Inspektors Kühn auf dem Dominium Kleutsch. Von den in unserem Besitz befindlichen Stücken passen vier mit vollkommen scharfen Bruchflächen aneinander, diese ergeben dann die in Fig. 4 dargestellte Form. Aber auch die übrigen Stücke lassen sich in ihrer Zugehörigkeit und Stellung zu diesem Teile soweit mit Sicherheit erkennen, dass man die ganze Form des Steins daraus rekonstruieren kann. Fig. 5 stellt diese dar, der links der Linie g f liegende Teil entspricht dem in Fig. 4 dargestellten Stücke. Besonders das die obere Endigung darstellende Stück ist in seiner Stellung an g ganz genau anzufügen, da hier die Kanten vollkommen aneinander passen. Aus der so ziemlich genau zu vollziehenden Ergänzung der fehlenden Teile erkennt man, dass von diesem Stücke fast die Hälfte in unsere Hand gelangte, und es kann darnach das Gesamtgewicht des Schobergrunder Steines nicht viel mehr als 1/2 Kilo betragen haben.

Die von uns rekonstruierte Form des Steines stimmt auch vollkommen mit den Angaben des Finders überein, dass er wie ein Keil ausgesehen habe. Auch an diesem Stücke ist die Orientierung deutlich. Die elliptische Grundfläche des Kegels, den Fig. 5 darstellt, ist die Brustfläche, auf ihr ist die Schmelzrinde auffallend dünn, so dass durch dieselbe das Gefüge des Steines vollkommen sichtbar bleibt. Außer auf ihr ist auf der in Fig. 4 abgewendet unter e liegenden Fläche gleichfalls die Schmelzrinde noch sehr dünn, so dass es den Anschein gewinnt, als sei die Kante e h Fig. 4 während des Fluges vorne gewesen. Auch an diesem Steine fehlen auf den anderen Seiten die fingerförmigen Eindrücke nicht.

Der kurze Durchmesser der elliptischen Basis h f misst 4 1/2 cm., der längere Durchmesser an dem vorhandenen Stücke 4 cm., also mit der Ergänzung etwa 8 cm. Die ganze Höhe bis zur ergänzten Spitze beträgt 9,5 cm.

Die Farbe der äußeren Schmelzrinde ist bei den beiden Steinen etwas verschieden. Bei dem Steine von Gnadenfrei ist sie überall vollkommen schwarz, während sie bei dem Steine von Schobergrund über die ganze Oberfläche hin rostfleckig erscheint; die Beschaffenheit der Rinde ist sonst dieselbe. Die Masse der beiden Steine ist nicht verschieden. In einer lichtgrauen Grundmasse, die außerordentlich bröcklich ist, liegen zahlreiche kleine Kugeln, die größten von etwa 2---3 mm. Durchmesser, die kleinsten nur wie winzige Punkte erscheinend. Die Farbe der Kugeln ist weiß, grün oder dunkelgrau. Neben ihnen erscheinen größere und kleinere Partien von metallischem Eisen, auf der Bruchfläche nur wenig hervortretend, aber auf einer angeschliffenen Fläche reichlicher sichtbar werdend. Mit der Lupe nimmt man außerdem kleinkörnige, bronzefarbige Parteien von Magnetkies und vereinzelte, messinggelbe Flitter von Troilit wahr.

Der Charakter achter Chondrit ist an beiden Steinen durch das besonders reichliche Vorhandensein der Kugeln sehr bestimmt ausgeprägt. In Fig. 5 an der vorderen Bruchfläche ist der Versuch gemacht, dieses darzustellen.

Der etwas abweichenden rostfleckigen Farbe der Schmelzrinde entspricht bei dem Steine von Schobergrund auch die Färbung des Innern. Auch die graue Grundmasse erscheint stellenweise ganz rostbraun geworden oder wenigstens mit zahlreichen Rostflecken bedeckt. Man nimmt wahr, dass diese Rostfarbe vorzüglich auf den Rissen, welche die Schmelzrinde durchziehen, in das Innere eingedrungen ist; im Innern der einzelnen Stücke tritt beim Durchschlagen die frische graue Farbe, wie sie der Gnadenfreier Stein besitzt, wieder hervor. Wenn man aber ein Stückchen des letzteren nur auf kurze Zeit ins Wasser taucht und dann liegen lässt, so wird es ebenfalls schon nach wenigen Stunden rostfleckig. Es wird diese Färbung also bewirkt durch eine außerordentlich schnelle Oxydation der metallischen Eisenteile. Auf diese schnelle Oxydation des meteorischen Nickeleisens hat auch schon (G. Rose aufmerksam gemacht\footnote{\frakfamily{Beschreibung und Einteilung der Meteoriten. Akad. Berlin 1863. p. 87.}}). Die Zeit eines halben Tages, die der Schobergrunder Stein im feuchten Ackerboden lag, war hinreichend, ihn so zu oxydieren.

An vielen der in der Grundmasse liegenden Kugeln war die auch von A. Makowsky und G. Tschermak an dem Meteorsteine von Tieschitz\footnote{\frakfamily{Denkschriften der mathem.-naturwiss. Klasse der Akademie. Wien. XXXIX. 1878. p. 11. Sep.-Abdr.}} beobachtete Erscheinung rundlicher Eindrücke wahrzunehmen. Uns scheinen diese Eindrücke von kleineren Kügelchen herzurühren, die bei der ursprünglichen Bildung den noch plastischen größeren sich anlagerten. Später wurden sie wieder auseinandergerissen. Jedenfalls sind diese Eindrücke auch nach unserer Auffassung ein Beweis, dass die mit ihnen versehenen Kugeln als klastische Bestandteile angesehen werden müssen. Auch Kugeln mit rundlichen, unregelmäßig höckerartigen Hervorragungen finden sich. Es lassen sich diese kaum mit etwas Anderem vergleichen als mit ähnlichen Formen an den Sphärolithen der trachytischen Gesteine. Wie bei diesen zeigt sich im Innern keinerlei Verschiedenheit der Struktur, keinerlei Grenze zwischen dem Höcker und der eigentlichen Kugel, die zusammen ein einziges Ganze bilden.

Die Untersuchung dargestellter Dünnschliffe unter dem Mikroskope ließ folgende Bestandteile der Gesteinsmasse erkennen: Nickeleisen, Magnetkies, Troilit, Chromeisen, Enstatit, Olivin und die aus diesen beiden Mineralien gebildeten Kugeln.

Das Nickeleisen bildet Partien von sehr verschiedener Größe, sehr unregelmäßiger Umgrenzung und zackig zerrissener Oberfläche (Fig. 6, a). Dort wo es die Kugeln oder andere Bestandteile umschließt, pflegt es mit vollkommen der Begrenzung jener entsprechendem, scharfem Rande sich an sie anzufügen. So erscheint es an vielen Stellen wie das Bindemittel, welches die übrigen nicht metallischen Teile zusammenhält. In kleineren Körnern, oft ziemlich regelmäßige Kugelform aufweisend, findet es sich aber auch in den nicht metallischen Teilen und den Kugeln verteilt und bildet jedenfalls auch einen Teil der schwarzen, staubförmigen Substanz, die in den Enstatit- und Olivinkörnern und Kugeln oft sehr dicht vorhanden ist. Alle Eisenpartien sind von einem ziemlich breiten rostfarbigen Saume umgeben, dessen Färbung sich den anliegenden Olivin- und Enstatitpartien mitteilt. Der lebhafte blaue, stahlartige Reflex lässt unter dem Mikroskope die Eisenteile immer deutlich von dem Magnetkies unterscheiden.

Dieser bildet nur kleine, körnig aussehende Aggregate, die einen bronzefarbigen Reflex geben (Fig. 6, b). Einzelne kleine Körnchen liegen auch im Innern der Silicate und des Nickeleisens. Ganz sparsam vorkommende winzige Körnchen mit auffallend lichtem, gelbem Reflex unter dem Mikroskope, dürfen für Troilit gehalten werden.

Das Chromeisen versteckt sich größtenteils unsichtbar unter dem Nickeleisen; in einigen Olivinquerschnitten sind schwarze, quadratische Einschlüsse wahrzunehmen (Fig. 6, i), die wohl kleine Oktaeder von Chromeisen sein mögen.

Der Enstatit erscheint sowohl als Bestandteil der grauen Gesteinsgrundmasse als auch in der Form isolierter Kugeln.

In der Grundmasse zwischen den Nickeleisenpartien bildet er meist verworren stenglige Aggregate, von weißer oder etwas gelblicher Farbe. Die einzelnen Stengel oder Leistchen sind ausgezeichnet durch eine sehr feine Längsstreifung und quer zu dieser hindurchsetzende Sprünge. (Fig. 6, k).

So erscheint er manchen terrestrischen Enstatitvorkommen außerordentlich ähnlich, ganz auffallend den Enstatitleisten, die in regelmäßiger Verwachsung mit Diallag in dem Basalte des Gröditzberges in Schlesien vorkommen und von Trippke beschrieben und vortrefflich abgebildet worden sind\footnote{\frakfamily{Zeitschr. d. deutsch. geol. Ges. 1878. XXX. p. 167. Taf. VIII.}}). Querschnitte eigentlicher, größerer Kristalle oder Kristallleisten sind sehr selten, zeigen dann aber die Beschaffenheit des Enstatits in besonders charakteristischer Weise. Fast alle haben eine etwas rudimentäre Form (Fig. 6, h). Immer zeigen diese Leisten eine der Längsfaserung parallele Orientierung der Auslöschungsrichtungen unter gekreuzten Nicols.

An Einschlüssen ist der Enstatit ziemlich reich, sie bestehen aus brauner oder farbloser Glasmasse, vielfach auch mit fixen Libellen und aus schwarzen metallischen Partikeln, sowie einer ebenfalls schwarzen nicht näher definierbaren staubförmigen Substanz, die zum Teil in wolkigen Anhäufungen erscheint. Alle Einschlüsse zeigen mehr oder weniger langgestreckte Formen in der Richtung der Faserung. In einigen Querschnitten ist besonders die staubförmige schwarze Substanz so dicht vorhanden, dass dieselben fast vollkommen undurchsichtig erscheinen. Überall aber charakterisieren neben der feinen Längsstreifung besonders die Querrisse auf das bestimmteste die Enstatitpartien.

Von den Kugeln bestehen die weißen größtenteils aus Enstatit. Die meisten erweisen sich im Dünnschliffe als ziemlich regellose Aggregate feingestreifter Stengel mit den charakteristischen Querrissen. In einer Kugel liegen größere und kleinere Leisten ohne irgend erkennbare Beziehung ihrer Stellung zur Kugelform durcheinander. Andere Kugeln zeigen aber auch eine bestimmte, regelmäßige Struktur. Sehr ausgezeichnet sind exzentrisch strahlige Kugeln, wie eine solche in Fig. 7 abgebildet ist. Die Fasern sind alle lange Leisten von Enstatit, ein Rand von sehr feinfasriger und durch zahlreiche schwarze Interpositionen verdunkelter Masse umschließt die Kugeln. Eine ähnliche Beschaffenheit besitzt die Kugel nach dem Zentrum der Fasern hin, so dass im Ganzen auch eine gewisse konzentrische Struktur hervortritt, deren Mittelpunkt aber gleichfalls zur Kugel selbst exzentrisch liegt. Solche exzentrisch fasrigen Kugeln sind früher schon mehrfach u.a. auch von G. Rose\footnote{\frakfamily{Meteoriten p. 98; Taf. IV. Fig. 8.}} und G. Tschermak\footnote{\frakfamily{l. c. p. 12. Fig. 7.}} beschrieben und abgebildet worden. Die von uns in Fig. 7 dargestellte Kugel umschließt einzelne Körnchen von Olivin. Bei dem fast in allen Fällen ausgesprochen rudimentären Aussehen dieser Kugelquerschnitte möchte man geneigt sein, sie nur für äußerlich wieder abgerundete Bruchstücke ursprünglich zentrisch radialfasriger Kugeln zu halten. Es kommen auch solche Kugeln von Enstatit in unseren Meteoriten vor, die eine, wenn auch nicht ganz regelmäßige, so doch bestimmt zentrisch radiale Gruppirung der Lamellen zeigen. Der Querschnitt einer solchen Kugel, wie er bei etwa 30 facher linearer Vergrößerung erscheint, ist in Fig. 10 dargestellt. Die Stellung der einzelnen Enstatitleisten ist im Zentrum ziemlich regelmäßig radial. Aber die der größeren Axe der etwas elliptisch geformten Kugel parallel liegenden Leisten biegen nach außen um und bilden beiderseitig Übergänge in die aus konzentrisch querliegenden Leisten gebildete schmale Randzone der Kugel. Aber im Großen und Ganzen ist doch das Bestreben nach zentrisch radialer Struktur unverkennbar.

Im Allgemeinen erscheint Olivin in den Enstatitkugeln nur spärlich in einzelnen Körnern zwischen die Enstatitleisten eingeklemmt. Auch finden sich größere Olivinquerschnitte rings von Enstatit umhüllt; dann erscheint die Art der Verwachsung beider Mineralien recht eigentümlich. Fig. 11 stellt das Rudiment einer Enstatitkugel dar, deren Inneres ein größerer Olivinkristall ein einnimmt, dessen Umrisse zum Teil noch sichtbar sind. In vielfachen Fetzen und Streifen greift der Enstatit von Außen in den Olivin hinein, so dass dieser am Rande wie ausgefranst erscheint durch die sich in ihn einschiebenden faserigen und querrissigen Lamellen; jede scharfe Grenze zwischen beiden verschwindet. Nur eine vollkommene Gleichzeitigkeit der Bildung beider lässt dieses Verhalten erklärbar erscheinen.

Verhältnismäßig selten sind Kugeln, in denen Olivin und Enstatit in so gleichmäßiger Verteilung vorhanden sind, dass die Entscheidung schwierig wird, eine solche Kugel dem einen oder andern Minerale zuzuweisen; fast immer überwiegt ein Bestandteil auf das Bestimmteste.

Der Olivin erscheint in der eigentlichen Grundmasse vielleicht etwas spärlicher als der Enstatit, dagegen sind die Olivinkugeln die häufigeren. In der Grundmasse erscheint er in der Form abgerundeter Körner oder Kristallbruchstücke von sehr unregelmäßigen zerrissenen Konturen, vielfach deutlich zerbrochen aussehend. Bruchstücke zertrümmerter größerer Körner liegen oft in noch erkennbarer Zusammengehörigkeit nahe beieinander. Selten sind in der Grundmasse scharfe, wohlerhaltene Kristallquerschnitte, die in den Kugeln außerordentlich häufig sich finden. Im Dünnschliffe erscheint der Olivin immer farblos, nur da, wie auch der Enstatit rostfarbig, wo er in der durch Oxydation des Nickeleisens gefärbten Zone liegt. Vom Enstatit unterscheidet ihn immer scharf auch in den kleinsten Partikeln das Fehlen der feinen Faserung in Verbindung mit den Querrissen. An Einschlüssen ist er noch reicher wie der Enstatit, obgleich auch fast vollkommen reine, einschlussfreie Querschnitte vorkommen. Er enthält ebenfalls braune Glaseinschlüsse, viele mit einer oder auch 2 fixen Libellen (Fig. 6, d), metallische Körner, Magnetkies, Chromeisen und die auch beim Enstatit angeführte schwarze, meist staubförmig erscheinende Substanz.

Alle äußerlich grünen Kugeln scheinen dem Olivin anzugehören. In Dünnschliffen sind sie ebenfalls farblos und zeigen eine ziemliche Verschiedenheit ihrer Struktur. Am häufigsten scheinen die Kugeln zu sein, die als ein Aggregat von Kristallen oder Kristallkörnern sich erweisen. In vielen Fällen sind sie vollkommen regellose Zusammenhäufungen von ziemlich gleich großen Olivinkörnern. Die einzelnen Körner, manchmal auch Kristallumrisse zeigend, erscheinen mit der schwarzen, staubförmigen Substanz oft dicht umrandet und diese erfüllt auch die Lücken zwischen den Körnern und gibt dadurch vor allem der ganzen Kugel einen ein heitlichen Ausdruck. Unter der wolkigen Hülle dieses schwarzen Staubes verschwimmen die Konturen der einzelnen Körner. Oft bildet das Zentrum einer solchen Kugel ein einziger oder mehrere größere Kristalle von vollkommen scharfen Umrissen, (Fig. 6, f und d) um die sich dann die kleineren Körner oder Kristalle herumlegen. Wo größere in einer solchen Kugel vereinigte Kristallquerschnitte am Rande derselben liegen (Fig. 8), bilden die geradlinigen Konturen jener immer sichtbare Unterbrechungen in der Rundung. So fand sich eine Kugel, die nur aus drei Kristallen besteht, die mit den spitzen Endigungen der Querschnitte aneinander gelegt sind, so dass ihre nach Außen gewendeten Längsseiten und die. diesen anliegenden Seiten der Endigung ein ziemlich regelmäßiges Hexagon bilden; die Lücken sind mit der schwarzen, staubförmigen Substanz erfüllt und auch die Kristalle davon umgeben. Solche Erscheinungen scheinen den Beweis zu liefern, dass die Bildung dieser Kugeln nicht eine Folge der Kristallisation ihrer Bestandteile, sondern lediglich der mechanischen Aggregation präexistirender Kristalle sein kann.

Die Gruppierung der kleineren Körner um einen größeren Kristall ist in den Kugeln aber oft auch recht regelmäßig. Sie umgeben denselben wie ein Kranz, dessen einzelne Glieder in ihren Konturen verschwimmen oder in einander überzugehen scheinen (Fig. 6, f). So umschließt ein aus solchen einzelnen Gliedern zusammengefügter Ring, fast wie aus einem einzigen Stück bestehend, einen größeren Kristall im Innern in Fig. 6, c. Aber unter gekreuzten Nicols tritt die Selbständigkeit der einzelnen Teile in der abweichenden Lage der Hauptschwingungsrichtungen bestimmt hervor. Unverkennbar ist an einigen Kugeln eine spiralige Anordnung der einzelnen Körner und Kristallquerschnitte, die sie zusammensetzen. Eine solche ist bei e, Fig. 6 dargestellt. Sie machen fast den Eindruck, als ob sie durch ein Aufrollen sich gebildet hätten.

Die dichtere Imprägnation mit der schwarzen, staubförmigen Substanz nach Außen bewirkt oft einen dunklen Rand der Kugeln, ebenso aber kommen Kugeln mit verdunkeltem Kerne vor. Die lediglich körnige Aggregation aber wird durch diese Imprägnation selten ganz verhüllt. Alle größeren im Innern der Kugeln liegenden Kristallquerschnitte sind in gleicher Weise von einer solchen nach Außen meist sehr unregelmäßig lappig verlaufenden, mehr oder weniger breiten Zone dieser verdunkelten oft ganz schwarzen und undurchsichtigen Substanz umgeben (Fig. 6, f u. d). Ein die einzelnen Kristalle oder Körner umhüllendes oder sie vereinigendes Bindemittel konnte nirgendwo in den Kugeln wahrgenommen werden.

Dem Olivin gehören dann auch die dunkelgrauen Kugeln an, die man auf der Bruchfläche der Stücke wahrnimmt und die besonders beim Pulvern des Gesteins durch ihre auffallend große Härte gegenüber den meist leicht zerbröckelnden übrigen Bestandteilen sich auszeichnen.

Diese grauen Kugeln sind immer von einer ganz eigenartigen Struktur. Sie entsprechen jenen, die schon G. Rose beschrieben und dargestellt hat.\footnote{\frakfamily{Meteoriten p. 95. Taf. IV. Fig. 8 u. 9.}} Das charakteristische dieser Kugeln wie es in Dünnschliffen hervortritt, besteht darin, dass sie größtenteils nur aus einem einzigen Individuum bestehen, wie das die einheitliche optische Orientierung zeigt, und dass sie in ihrer ganzen Masse von quer durch die Kugel hindurchsetzenden parallelen Streifen von unbestimmt grauer Farbe, die in der einheitlich polarisierenden Masse liegen, durchsetzt werden (Fig. 6, g). Gleichzeitig erscheint die Kugel fast gleichmäßig von punktförmigen oder auch größeren Einschlüssen metallischer und schwarzer Partikel erfüllt, die wesentlich die graue Farbe der ganzen Kugel bedingen (Fig. 9). Wo eine etwas dünnere oder klare Stelle im Innern eines solchen Kugelquerschnittes die Anwendung stärkerer Vergrößerung gestattet, zeigt sich, dass die dunkelgrauen Streifen größtenteils dadurch bewirkt werden, dass zwischen stabförmig oder stenglich ausgebildeten Gliedern einer solchen Olivinkugel feine Risse, oder auch unregelmäßig, schlauchförmige Öffnungen übrig geblieben sind (Fig. 9, d) auf denen sich dann die schwarze Substanz angehäuft hat. Diese sehr unregelmäßigen Risse liegen in einer Platte zu vielen übereinander und rufen im Mikroskope den Eindruck der grauen Streifen hervor, zwischen denen dann nur schmale Streifen heller Substanz übrig bleiben. Jedenfalls sind die Risse mehr beteiligt an der Erscheinung der grauen Streifen, als die schwarzen Interpositionen, wenngleich diese dazu beitragen, die Streifung noch schärfer auszuprägen.

Stets liegen die Streifen so, dass sie mit der einen Auslöschungsrichtung im Olivin zusammenfallen. Es können daher die einzelnen, unregelmäßig stabförmigen Glieder, in welche eine solche Kugel sich zerlegt, als nach der Hauptaxe gestreckte Mikrolithe angesehen werden. Es liegt hier eine feinere Ausbildung derselben Struktur vor, wie sie manche Olivinkugeln in den Meteoriten von Pultusk in ausgezeichneter Weise zeigen. Diese Kugeln erscheinen gleichfalls von mit schwarzer Substanz erfüllten, wellig verlaufenden Sprüngen von nahezu paralleler Stellung durchsetzt, die sich vielfach vereinigen oder durch Querrisse verbunden sind, und mehr oder weniger länglich-rundliche, klare Olivinpartien zwischen sich lassen. Das ganze Bild zeigt ein Maschenwerk, das sich am zutreffendsten mit der Struktur der Bienenwaben vergleichen lässt. Die ganze Kugel zeigt durchaus einheitliche Polarisationserscheinung und zwar liegt auch hier die Auslöschungsrichtung immer parallel dem Streifensysteme. Für diese Ausbildung der Olivinkugeln dürfte es wohl zweierlei Erklärung geben. Einmal könnte es eine ähnliche Bildung sein, wie wir sie an terrestrischen Mineralien z. B. der Hornblende kennen, wo sich häufig viele parallel stehende dünne Stengel oder langgestreckte Mikrolithen zu einem größeren Individuum vereinigen. Wahrscheinlicher aber dürfte die andere Deutung sein, dass ein ursprünglich einfacher, klarer Olivinkristall durch plötzliche Erhitzung oder schnell folgende Abkühlung dazu, in seiner ganzen Masse rissig geworden und dass die Risse die wenn auch unvollkommene Spaltbarkeit des Olivins parallel $\infty P \infty$ in vollkommenerer Weise zum Ausdruck bringen, als dieses gewöhnlich wahrzunehmen ist, etwa so wie man auch beim Bergkristall, wenn man ihn heftig erhitzt und dann schnell abkühlt, eine der rhomboedrischen Spaltbarkeit entsprechende Richtung der entstehenden Risse wahrnehmen kann. Das Eindringen schwarzer Substanz auf den entstandenen Rissen in die Olivinkristalle ist dann erst später erfolgt, ebenso wie sie wohl erst später ihre Kugelform erhalten haben mögen.

Die gestreiften Olivinkugeln haben auf den ersten Blick eine gewisse Ähnlichkeit mit Enstatitkugeln; die genau parallele Stellung der Streifen, das Fehlen der feinen Längsstreifung und der Querrisse und vor Allem die einheitliche Polarisationserscheinung der ganzen Masse unterscheidet sie immer auf das schärfste.

Am eigentümlichsten sind solche Kugeln, die ihrer ganzen Beschaffenheit nach durchaus als primäre Kugeln angesehen werden müssen, in denen aber die Streifung in den einzelnen Teilen der Kugel in verschiedenen Richtungen verläuft. Eine solche ist in Fig. 9 dargestellt. Die Streifensysteme grenzen zum Teil vollkommen scharf und geradlinig aneinander, sie gehen aber auch durch eine Knickung, die eine scharfe Grenze nicht ergibt, in eine andere Richtung über (Fig. 9 oben). In den einzelnen Teilen dieser Kugel ist die Richtung der Auslöschung immer parallel dem Streifensysteme. An einer anderen Kugel sind nur zwei Streifensysteme vorhanden, die gerade im Durchmesser der Kugel geradlinig aneinanderstoßen und einen Winkel von 130° mit einander bilden. Man könnte hierbei fast an eine Zwillingsbildung denken. Jedenfalls sind diese Kugeln nicht erst aus der Vereinigung von Teilen älterer, zertrümmerter Kugeln entstanden; ihre ganze Erscheinung spricht mit Bestimmtheit dafür, dass es primäre, einheitlich gebildete Kugeln sind.

Die in diesen Olivinkugeln liegenden staubförmigen Interpositionen sind oft so dicht gedrängt, dass auch in dünnen Schliffen die Streifung kaum mehr sichtbar bleibt. In andern Kugeln ist sie in Wirklichkeit nicht vorhanden und diese erscheinen bei schwacher Vergrößerung als einfache dunkelgraue Scheiben. Aber die einheitliche Orientierung der Auslöschung lässt sich am Rande, dort wo diese Kugeln sehr dünn und durchscheinend werden, dennoch erkennen.

Der Gesamteindruck, den im Dünnschliffe die Struktur unserer Meteorit macht, ist entschieden der eines Trümmergesteins. Besonders sind es außer den Bestandteilen der eigentlichen Grundmasse die vielen Kugelrudimente, die diesen Eindruck hervorrufen. Solche zerbrochene oder halbe Kugeln sind nicht selten; in Fig. 6 bei g und c sind solche dargestellt. Solche, an denen der äußere Rand nicht mehr scharf, sondern wie beschädigt erscheint, sind noch häufiger (Fig. 11). Auch die Bestandteile der Grundmasse sind nicht selten so aggregiert, dass man die Reste zertrümmerter kugliger Gebilde darin noch erkennen kann. Die außerordentlich bröckliche Beschaffenheit der Grundmasse darf wohl auch auf ihre klastische Struktur zurückgeführt werden, zumal ein Bindemittel außer etwa dem metallischen Eisen, nirgendwo in derselben wahrgenommen werden kann. ---

Beim Pulvern der zur Analyse zu verwendenden Menge. des Meteoriten (es diente hierzu ein Stückchen des Gnadenfreier Steines) erwiesen sich die reichlich vorhandenen Eisenteile als hinderlich. Sie wurden daher immer mit dem Magneten aus dem groben Pulver entfernt, dann der Rest feingepulvert, die ausgezogenen Eisenteile für sich noch zerkleinert und dem Ganzen wieder zugefügt. Nur so gelang es, das Silicat zur Analyse hinlänglich fein gepulvert zu erhalten.

Die Analyse wurde in der Weise ausgeführt, dass das gepulverte Material unter Abschluss der Luft in einer Kohlensäureatmosphäre mit Kupferchloridlösung behandelt und hierdurch Eisen und Nickel ausgezogen\footnote{\frakfamily{Vergl. Sipocz. Tschermaks mineral. Mitteilungen 1874. p. 244.}} und nach Entfernung des Kupfers quantitativ bestimmt wurden. Mit einem Teile des Silicatrestes wurde die Bestimmung der Alkalien ausgeführt, der Rest mit Salzsäure behandelt und der unlösliche Bestandteil nach Entfernung der freien Kieselsäure bestimmt. In einer zweiten abgewogenen Menge wurde mit dem Magneten sorgfältigste alle magnetischen Bestandteile ausgezogen und dann der Silicatrest nochmals in seinem löslichen und unlöslichen Bestandteil analysiert. Von den doppelt erhaltenen Werten sind die Mittel genommen worden. Zur Bestimmung des Schwefels wurde eine dritte Quantität mit Salpetersäure oxydiert und der Schwefel als schwefelsaurer Baryt bestimmt. Nach einer bezüglichen Korrektur, wodurch alle Tonerde, Kalk und Natron, die zum Teil mit in Lösung gegangen, wieder auf den unlöslichen Teil umgerechnet wurden, ergaben sich dann die Resultate der Analyse wie folgt:
\clearpage
\begin{center}
Gesammtanalyse:
\end{center}
\begin{center}
\begin{tabular}{ l r }
    SiO$_{2}$ & 32,11\\
    Al$_{2}$O$_{3}$ & 1,60\\
    FeO & 14,88\\
    MgO & 17,03\\
    CaO & 2,01\\
    Na$_{2}$O & 0,70\\
    Fe & 25,16\\
    Ni & 3,92\\
    S & 1,87\\
    Cr$_{2}$O$_{3}$ & 0,57\\
    PO$_{5}$, MnO, Co & Spuren, nicht bestimmt.\\
     & 99,85
\end{tabular}
\end{center}
\paragraph{}
Das specif. Gewicht ergab bei 16° C. von drei verschiedenen Proben: 3,644, 3,712 und 3,785.
\begin{center}
Silicatanalyse:
\end{center}
\paragraph{}
A. Lösliches Silicat = Olivin: 34,02 pCt. des Ganzen berechnet, gefunden = 35,01 pCt.

B. Unlösliches Silicat = eisenreicher Enstatit; 34,03 pCt. des Ganzen berechnet, gefunden --- 33,23 pCt.

C. Zusammensetzung des Silicates als Summe des löslichen und unlöslichen Teiles berechnet.

D. Berechnete Zusammensetzung des Silicates nach dem Verhältnis von 34,02 Olivin : 34,03 Enstatit.
\begin{center}
\begin{tabular}{ l r r r r }
     & A. & B. & C. & D.\\
    SiO$_{2}$ & 17,20 & 29,63 & 46,83 & 46,38\\
    Al$_{2}$O$_{3}$ & - & 2,34 & 2,34 & 2,08\\
    FeO & 12,16 & 9,12 & 21,28 & 22,30\\
    MgO & 13,43 & 11,43 & 24,86 & 25,83\\
    CaO & - & 2,83 & 2,83 & 2,51\\
    Na$_{2}$O & - & 1,02 & 1,02 & 0,90\\
    Summen & 42,79 & 56,37 & 99,16 & 100,00\\
\end{tabular}
\end{center}
\clearpage
\paragraph{}
Es besteht sonach die Masse des Meteoriten aus:
\begin{center}
\begin{tabular}{ p{25mm} r p{35mm} r }
    Fe & 22,34 & Nickeleisen & 26,16\\
    Ni & 3,92 & Nickeleisen & 26,16\\
    Fe & 2,92& Fe$_{8}$S$_{9}$ Magnetkies (und Troilit) & 4,79\\
    S & 1,87& Fe$_{8}$S$_{9}$ Magnetkies (und Troilit) & 4,79\\
    FeO & 0,28 & Chromeisen & 0,85\\
    Cr$_{2}$O$_{3}$ & 0,57 & Chromeisen & 0,85\\
    Unlösl. Silicat (Enstatit) & 34,03 & Silicat & 68,05\\
    Lösl. Silicat (Olivin) & 34,02 & Silicat & 68,05\\
     & 99,85 & & 99,85 pCt.\\
\end{tabular}
\end{center}
\paragraph{}
Das Nickeleisen besteht aus:
\begin{center}
\begin{tabular}{ l r }
    Fe & 85,1 pCt.\\
    Ni & 14,9\\
     & 100,00\\
\end{tabular}
\end{center}
und entspricht sonach ganz nahe der Formel Fe$_{6}$Ni. Die Phosphorsäure deutet auf eine im Nickeleisen enthaltene geringe Menge von Schreibersit.
\paragraph{}
Der Enstatit hat die aus der Analyse des unlöslichen Teiles B berechnete Zusammensetzung:
\begin{center}
\begin{tabular}{ l r }
    SiO$_{2}$ & 52,56\\
    Al$_{2}$O$_{3}$ & 4,15\\
    FeO & 16,18\\
    MgO & 20,28\\
    CaO & 5,02\\
    Na$_{2}$O & 1,81\\
     & 100,00\\
\end{tabular}
\end{center}
\paragraph{}
Er ist ein tonerdehaltiger Enstatit, ähnlich dem aus dem Meteoriten von Chantonnay, sowie dem von Hainholz; seine Zusammensetzung entspricht der Mischung von
\begin{center}
3 Mg SiO$_{3}$
\end{center}
\begin{center}
Fe SiO$_{3}$.
\end{center}
\paragraph{}
Der Olivin hat die aus der Analyse berechnete Zusammensetzung:
\begin{center}
\begin{tabular}{ l r }
    SiO$_{2}$ & 40,20\\
    FeO & 28,42\\
    MgO & 31,38\\
     & 100,00\\
\end{tabular}
\end{center}
\paragraph{}
Er entspricht sonach ganz nahe der Mischungsformel
\begin{center}
2 Mg$_{2}$ SiO$_{4}$
\end{center}
\begin{center}
Fe$_{2}$ SiO$_{4}$
\end{center}
wie der Olivin in dem Meteoriten von Chassigny u. A.
\paragraph{}
Bezüglich des Quantitätsverhältnisses der einzelnen Bestandteile befindet sich die mikroskopische und chemische Analyse in vollkommener Übereinstimmung.
\clearpage
\section{\frakfamily{Erkl"arung der Tafeln.}}
\paragraph{}
Fig. 1 --- Der Stein von Gnadenfrei. In Fig. 3 ist die nach oben gewendete Kläche zwischen a und k die Brustfläche. Über die Stellung der einzelnen Figuren orientiert die jedesmalige Lage der Kante a b. In Fig. 1 und 2 fehlen die beiden abgetrennten Stücke, die in Fig. 3 links bei kl und ki angefügt sind. Die Zeichnungen sind in halber natürlicher (Größe ausgeführt. Über der Ecke d in Fig. 2 erscheinen auf der Rückfläche ausgezeichnet die fingerförmigen Eindrücke.

Fig. 2 --- Der Stein von Gnadenfrei. In Fig. 3 ist die nach oben gewendete Kläche zwischen a und k die Brustfläche. Über die Stellung der einzelnen Figuren orientiert die jedesmalige Lage der Kante a b. In Fig. 1 und 2 fehlen die beiden abgetrennten Stücke, die in Fig. 3 links bei kl und ki angefügt sind. Die Zeichnungen sind in halber natürlicher (Größe ausgeführt. Über der Ecke d in Fig. 2 erscheinen auf der Rückfläche ausgezeichnet die fingerförmigen Eindrücke.

Fig. 3 --- Der Stein von Gnadenfrei. In Fig. 3 ist die nach oben gewendete Kläche zwischen a und k die Brustfläche. Über die Stellung der einzelnen Figuren orientiert die jedesmalige Lage der Kante a b. In Fig. 1 und 2 fehlen die beiden abgetrennten Stücke, die in Fig. 3 links bei kl und ki angefügt sind. Die Zeichnungen sind in halber natürlicher (Größe ausgeführt. Über der Ecke d in Fig. 2 erscheinen auf der Rückfläche ausgezeichnet die fingerförmigen Eindrücke.

Fig. 4 --- Der Schobergrunder Stein. Fig. 4 stellt die aus vier mit den Bruchflächen noch scharf aneinander passenden Stücken zusammengefügte Hälfte des Steines dar. Fig. 5 zeigt die nach den übrigen noch vorhandenen Stücken rekonstruierte wahrscheinliche Form des ganzen Steines. Größe 2/3 der natürlichen.

Fig. 5 --- Der Schobergrunder Stein. Fig. 4 stellt die aus vier mit den Bruchflächen noch scharf aneinander passenden Stücken zusammengefügte Hälfte des Steines dar. Fig. 5 zeigt die nach den übrigen noch vorhandenen Stücken rekonstruierte wahrscheinliche Form des ganzen Steines. Größe 2/3 der natürlichen.

Fig. 6 --- Partie aus einem Dünnschliffe des Meteoriten. a. Nickeleisen mit zackiger Oberfläche. b. Magnetkies, körnig, mit bronzefarbigem Reflex. c. Olivinkugel, im Innern ein großer Kristall, mit einem Rand von kleineren Kristallen. d. Olivinkugel, zwei scharfe Kristallquerschnitte im Innern, körnige mit schwarzer, staubförmiger Substanz erfüllte Olivinmasse darum. e. Spiralig aufgerollte aus Olivinkörnern bestehende Kugel. f. Olivinkugel mit einem größeren, scharf konturierten Kristall im Innern, der sehr stark von der schwarzen, staubförmigen Substanz umrandet ist. g. Zerbrochene, gestreifte Olivinkugel, optisch einheitlich sich verhaltend, Auslöschungsrichtung parallel den Streifen. h. Querschnitt eines isolierten Enstatitkristalles, feine Längsstreifung mit Querrissen. i. Chromeisenoktaöder in Olivin. k. Enstatit als Grundmasse.

Fig. 7 --- Exzentrisch, fasrige Enstatitkugel mit zwei körnigen Olivineinschlüssen.

Fig. 8 --- Kugliges Aggregat von Olivinkristallen.

Fig. 9 --- Gestreifte Kugel von Olivin; mehrere Streifensysteme, jedesmal die Auslöschungsrichtung den Streifen parallel. Einschlüsse von Nickeleisen und staubförmiger Substanz. Fig. 9 b ein Teil der Streifen bei starker Vergrößerung.

Fig. 9b --- Gestreifte Kugel von Olivin; mehrere Streifensysteme, jedesmal die Auslöschungsrichtung den Streifen parallel. Einschlüsse von Nickeleisen und staubförmiger Substanz. Fig. 9 b ein Teil der Streifen bei starker Vergrößerung.

Fig. 10 --- Zentrisch, radiale Kugel von stengligem Enstatit.

Fig. 11 --- Kugelrudiment von Enstatit im Innern einen größeren Olivinkristall umhüllend.
\clearpage
\setlength\intextsep{0pt}
\pagestyle{fancy}
\fancyhf{}
\rhead{\frakfamily{Tafel 1.}}
\cfoot{\thepage}
\begin{figure}[p]
\includegraphics[scale=0.9,keepaspectratio]{Figures/Table1/fig1.png}\tiny 1
\includegraphics[scale=0.9,keepaspectratio]{Figures/Table1/fig2.png}\tiny 2
\includegraphics[scale=0.9,keepaspectratio]{Figures/Table1/fig3.png}\tiny 3
\includegraphics[scale=0.9,keepaspectratio]{Figures/Table1/fig4.png}\tiny 4
\includegraphics[scale=0.9,keepaspectratio]{Figures/Table1/fig5.png}\tiny 5
\end{figure}
\clearpage
\begin{figure}[p]
\includegraphics[scale=1.5,keepaspectratio]{Figures/Table1/fig6.png}\tiny 6
\includegraphics[scale=1.2,keepaspectratio]{Figures/Table1/fig7.png}\tiny 7
\includegraphics[scale=1.3,keepaspectratio]{Figures/Table1/fig8.png}\tiny 8
\includegraphics[scale=1.2,keepaspectratio]{Figures/Table1/fig9.png}\tiny 9
\includegraphics[scale=1.2,keepaspectratio]{Figures/Table1/fig9b.png}\tiny 9b
\includegraphics[scale=1.2,keepaspectratio]{Figures/Table1/fig10.png}\tiny 10
\includegraphics[scale=1.2,keepaspectratio]{Figures/Table1/fig11.png}\tiny 11
\end{figure}
\end{document}
